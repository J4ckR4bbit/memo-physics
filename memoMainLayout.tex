\documentclass{article}

\usepackage{amsmath}
\usepackage{tikz, pgfplots}
\usetikzlibrary{arrows,
    calc,
    decorations,
    scopes,
}

\begin{document}

\section*{THE STATIC COEFFICIENT OF FRICTION}
\subsection*{DEFINITINONS}
\begin{quote}
The normal force $F_N$
The perpendicular force exerted by a surface on an object in contact with it.
\begin{equation}
F_N = F_{g\perp} = F_g cos \theta  
\end{equation}
Frictional Froce $F f$ \\
The force that opposes the motion of an object and acts parallel to the surface with which the object is in contact.
\begin{equation}
F_{f(max)} = F_{g\parallel} = F_g sin \theta
\end{equation}
\end{quote}

\subsection*{}

    \def\iangle{20} % Angle of the inclined plane
    \def\down{-90}
    \def\arcr{0.5cm} % Radius of the arc used to indicate angles
\begin{center}
    \begin{tikzpicture}[
        >=latex',
        scale=1,
        force/.style={->,draw=blue,fill=blue},
        axis/.style={densely dashed,gray,font=\small},
        M/.style={rectangle,draw,fill=lightgray,minimum size=0.5cm,thin},
        m/.style={rectangle,draw=black,fill=lightgray,minimum size=0.3cm,thin},
        plane/.style={draw=black,fill=blue!10},
        string/.style={draw=red, thick},
        pulley/.style={thick},
        ]
        \pgfmathsetmacro{\Fnorme}{1}
        \pgfmathsetmacro{\Fangle}{45}
        \begin{scope}[rotate=\iangle]
            \node[M,transform shape] (M) {};
            \coordinate (xmin) at ($(M.south west)-({abs(1.1*\Fnorme*sin(-\Fangle))},0)$);
            \coordinate (xmax) at ($(M.south east)+({abs(1.1*\Fnorme*sin(-\Fangle))},0)$);
            \coordinate (ymax) at ($(M.north)+(0, {abs(1.1*\Fnorme*cos(-\Fangle))})$);
            \coordinate (ymin) at ($(M.south)-(0, 1cm)$);
            \draw[postaction={decorate, decoration={border, segment length=3pt, angle=-45},draw,black}] (xmin) -- (xmax);
            \coordinate (N) at ($(M.center)+(0,{\Fnorme*cos(-\Fangle)})$);
            \coordinate (Fperp) at ($(M.center)-(0,{\Fnorme*cos(\Fangle)})$);
            \coordinate (fpara) at ($(M.center)+({\Fnorme*sin(-\Fangle)}, 0)$);
            \coordinate (fs) at ($(M.center)+({\Fnorme*sin(\Fangle)}, 0)$);
           
            % Forces
            \draw [->, thick, red](M.center) -- (N) node [above] {\tiny$\vec F_{N} = F_{g\perp}$};
            \draw [->, thick, red](M.center) -- (Fperp) node [below,pos=1.8 , right,pos=1.1 ] {\tiny$F_{g\perp} = F_g cos \theta$};
            \draw [->, thick](M.center) -- (fpara) node [left] {\tiny$F_{g\parallel} = F_g sin \theta$};
            \draw [->, thick](M.center) -- (fs) node [right] {\tiny$\vec F_{s(max)} = F_{g\parallel} $};
           
           % \draw (M.center)+(90+\Fangle:\arcr) arc [start angle=90+\Fangle,end angle=90,radius=\arcr] node [above, pos=.5] {\tiny$\theta=\Fangle$};
        \end{scope}
        % Draw gravity force. The code is put outside the rotated
        % scope for simplicity. No need to do any angle calculations. 
        \draw[force,->] (M.center) -- ++(0,-1) node[below, pos=1] {\tiny$\vec F_g$};
        \draw (M.center)+(-90:\arcr) arc [start angle=-90,end angle=\iangle-90,radius=\arcr] node [right] {\tiny$\theta$};
    \end{tikzpicture}
    \end{center}
    
    \subsection*{STATIC FRICTION}
    \begin{quote}
        The static coefficient of friction is the point just before an object starts to slide.\\
        At this point the two forces acting on the parallel plane are equal to zero.
     \begin{equation}
         \vec{F} = ma
     \end{equation}
     \begin{equation}
        F_{g\perp} - F_{s(max)} = 0
    \end{equation}        
        This can only occur at a specific angle in accordance with the coefficient of static friction.
        \begin{equation}
           F_{s(max)} = \mu_{s(max)} F_N 
        \end{equation}
        
        At any angle below this the object will not move so $F_{net} $ and formula (5) no longer applies, relevant formula will therefore be formula (4) as friction $F_{f}$ will have to be equal to  $F_{g\perp}$ \\
        $$F_{g\perp} - 0 =  F_{s}$$
        
        If the sum of $F_{g \perp}$ and $F_f$ does not equal zero we are no longer dealing with $\mu_{s}$ or $\mu_{s(max)}$, this is purely due to definition.\\
        
    \subsection*{IMPLICATIONS}
        In conclusion,This means in equation $(5)$\\ 
        The coefficient of friction $\mu_{s(max)}$ is a constant. \\
        \\
        The normal force $F_N$ will be the independent variable, as the angle of slope can be changed.   \\
        \\
        The frictional force $F_f$  will then then be the dependant variable, if the object is at rest it will always be equal to $F_{g \perp}$. \\
        \\
        The only time $F_f$ will not be equal to $F_{g \perp}$ is if the object is moving in this case the coefficient will change (usually lower than $\mu_{s(max)}$) This is called $\mu_{kinetic}$ and is not tested by the IEB.
    
   
          

        
        
    \end{quote}
    \end{document}